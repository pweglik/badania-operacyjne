% !TeX spellcheck = pl_PL
\documentclass[12pt,a4paper]{article}

% moje działają
\usepackage{amssymb} % symbol kąta
\usepackage[polish]{babel} % polskie nazwy
\usepackage[T1]{fontenc} % polskie znaki
\usepackage[margin=1.0in]{geometry} % marginesy
\usepackage[utf8]{inputenc}
\usepackage{listingsutf8} % bloki kodu
\usepackage{color} % kolory
\usepackage{indentfirst} % wcięcie w pierwszej linii paragrafu
\usepackage{graphicx} % obrazy
\usepackage{float} % dla image [H]
\usepackage{amsmath,amsthm,amssymb,mathtools} % matematyka dowód
\usepackage{changepage} % matematyka dowód
\usepackage{siunitx} % wyrównanie do kropki
\usepackage{makecell} % wyrównania nagłówków
%\usepackage{enumitem} % wyrównania nagłówków
\usepackage{tikz} % zbocza
\usetikzlibrary{decorations.markings}
\usepackage{hyperref} % bez obwódek wokół linków
\usepackage{algorithm}
\usepackage{algpseudocode} % pseudokod

% algorytmy po polsku
\floatname{algorithm}{Algorytm}
\floatname{required}{W}
\renewcommand{\listalgorithmname}{Spis algorytmów}
%cmds
\algnewcommand\algorithmicto{\textbf{to}}
\algnewcommand\algorithmicand{\textbf{and }}
\algnewcommand\algorithmicor{\textbf{or }}
\algnewcommand\algorithmictrue{\textbf{true}}
\algnewcommand\algorithmicfalse{\textbf{false}}
\algdef{S}[FOR]{ForTo}[3]{\algorithmicfor \  $ #1 \gets #2 $ \algorithmicto \ $ #3 $ \algorithmicdo}

\hypersetup{
    colorlinks,
    citecolor=black,
    filecolor=black,
    linkcolor=black,
    urlcolor=black
}

% obrazek {nazwa.png}{opis}
\graphicspath{ {./img/} }
\newcommand{\imgcustomsize}[3]{
	\begin{figure}[H]
		\centering
		\includegraphics[width=#3\textwidth]{#1}
		\caption{#2}
	\end{figure}
}
\newcommand{\img}[2]{\imgcustomsize{#1}{#2}{0.8}}

% dwa obrazki
% \imgsidebyside{1}{opis 1}{2}{opis 2}
\newcommand{\imgsidebyside}[4]{
	\begin{figure}[H]
		\centering
		\begin{minipage}{.45\textwidth}
			\centering
			\includegraphics[width=1\linewidth]{#1}
			\caption{#2}
		\end{minipage}%
		\hfill
		\begin{minipage}{.45\textwidth}
			\centering
			\includegraphics[width=1\linewidth]{#3}
			\caption{#4}
		\end{minipage}
	\end{figure}
}

\newtheorem{definition}{Def}

\begin{document}

\title{
    Badania operacyjne\\
    Projekt\\
}

\author{\\Jakub Kosmydel\\Norbert Morawski
    \\Przemysław Węglik\\Bartłomiej Wiśniewski}

\date{\today}

\maketitle
\newpage
\tableofcontents

\newpage

\section{Wstęp}
Celem naszego projektu jest znalezienie optymalnych tras linii dla autobusów, aby maksymalizować liczbę pasażerów, przy minimalizacji kosztów. Aby to osiągnąć, wykorzystywane są algorytmy genetyczne -  algorytmy przeszukujące przestrzeń rozwiązań, które opierają się na procesie działania mechanizmu dziedziczenia biologicznego.


Pozycje przystanków są ustawione na sztywno w miejscach inspirowanych ich prawdziwą obecnością w Krakowie, a ich popularność jest zależna od gęstości zaludnienia dookoła nich. Stosowanie algorytmów genetycznych pozwoliło na wygenerowanie zestawu najlepszych połączeń autobusowych, które można skonfigurować dla lepszego wykorzystania zasobów oraz zwiększenie korzyści z transportu publicznego dla pasażerów.

\section{Opis zagadnienia}

\subsection{Sformułowanie problemu}
Naszym celem w projekcie jest zaprojektowanie sieci linii autobusowych pokrywającej dany obszar miejski, przy danym rozłożeniu przystanków. Linie te, powinny mieć możliwość obsłużenia jak największej liczby pasażerów, tworząc jak najmniej postojów oraz zatrzymując się na jak najmniejszej liczbie przystanków.

\subsection{Model matematyczny}

\subsubsection{Założenia}

\begin{enumerate}
    \item Przystankom przypisujemy ilość punktów w zależności od gęstości zaludnienia w pobliżu.
        %   \begin{itemize}
        %       \item Dla każdego przystanku obliczamy liczbę ludzi w pobliżu,
        %       \item Głównym punktom w Krakowie (np. D17, teatry, itp.) nadajemy wartość punktową,
        %       \item Dla każdego przystanku sumujemy powyższe wartości.
        %   \end{itemize}
    \item Rozkładamy linie komunikacyjne po mieście tak, by maksymalizować sumę zebranych punktów przez wszystkie linie.
    \item Wprowadzamy koszt dla linii: koszt ścieżki w grafie, po której jedzie + koszt utworzenia nowej linii.
    \item Linie przebiegające przez jeden przystanek uzyskają więcej punktów niż jedna linia, ale stosujemy prawo malejących przychodów.
    \item Maksymalizujemy sumę punktów zebranych przez wszystkie linie.
\end{enumerate}

\subsubsection{Dane}
\begin{enumerate}
    \item $n$ - liczba linii
    \item $m$ - liczba przystanków
\end{enumerate}

\subsubsection{Graf}

\begin{enumerate}
    \item Wierzchołki to istniejące przystanki z przypisanymi punktami, zależącymi od gęstości zaludnienia w pobliżu.
    \item $p(j)$ - wartość punktowa przystanku:
          \begin{itemize}
              \item $p(j) = \sum_{i=0}^{n-1} \left[ w_{j, i} \cdot f(d_{j, i}) \right]$ gdzie $w_{j, i}$ to wartość obiektu (np. liczba mieszkańców w pobliżu) a $d_{j,i}$ to odległość tego bloku od przystanku, f – funkcja malejących zysków.
              \item Funkcja liczona dla danego przystanku $j$
          \end{itemize}
    \item Krawędzie to połączenia między przystankami.
    \item Koszt krawędzi to odległości między przystankami.
\end{enumerate}










\subsubsection{Szukane}
$x_{i,j}$ - czy linia $i$ zatrzymuje się na przystanku $j$, gdzie:
\begin{enumerate}
    \item $i \in \left[ 0, n-1 \right]$
    \item $j \in \left[ 0, m-1 \right]$
\end{enumerate}

\subsubsection{Hiperparametry}
\begin{enumerate}
    \item $\alpha$ - koszt zatrzymania się na przystanku,
    \item $\beta$ - koszt nowej linii,
    \item $K$ - funkcja dopasowania dla długości linii,
    \item $\Delta$ - koszt nieodwiedzenia przystanku,
    \item $R$ - hiperparametr zbiegania.
\end{enumerate}

\subsubsection{Funkcja zysku}
\begin{align*}
    l_j                         & = \sum_{i=0}^{n-1} x_{i, j}                                               & \text{liczba linii zatrzymujących się na przystanku $j$}                 \\
    q_j                         & = \frac{p_j \cdot (1+\frac{R}{l_j})^{l_j}}{l_j}                           & \text{ile punktów każda linia uzyskuje z przystanku $j$}                 \\
    \lim_{l_{j}\to\infty} q_{j} & =  \frac{e^R}{l_j}                                                        & \text{$q_{j}$ jest ograniczone nawet jeśli liczba lini jest bardzo duża} \\
    fitness_{j}                    & = \begin{cases}
        \sum_{i=0}^{n-1} x_{i,j} \cdot (q_j-\alpha) & l_j > 0 \\
        -\Delta                                     & l_j = 0 \\
    \end{cases}                                               & \text{zysk jednej lini  i penalizacja nieodwiedzonych przystanków}                           \\
    S_{i}                       &                                                                           & \text{długość ścieżki linii $i$ w grafie}                                \\
    f(x)                        & = \sum_{j=0}^{m-1} cost_{j} - \sum_{i=0}^{n-1} \left[ K(S_{i})-\beta \right] & \text{funkcja zysku}                                                    \\
\end{align*}
gdzie
\begin{itemize}
	\item $K$ -- funkcja skalująca długość, która dobrana odpowiednio pozwala uniknąć patologicznych sytuacji linii długości tylko 2 lub bardzo długich linii.
\end{itemize}

W generacji linii dla Krakowa użyto funkcji $K(s) = T - C \, (s - E)^2$, gdzie parametry przyjęto następująco:
\begin{itemize}
	\item $T=20000$ (top, wartość maksymalna funkcji)
	\item $C=5$ (cutoff, współczynnik zmniejszania się funkcji)
	\item $E=10$ (expected, "wartość oczekiwana", funkcja przyjmuje w $s=E$ maksimum)
\end{itemize}

Względem oryginalnego modelu zmieniło się:
\begin{itemize}
	\item Dodano funkcje $K$
\end{itemize}







\section{Opis algorytmów}
Nasz problem rozwiązywaliśmy algorytmami genetycznymi.

\subsection{Reprezentacja środowiska}
Jak już zostało wspomniane, zajmowaliśmy się problemem optymalizacji istniejącej sieci komunikacyjnej, bez tworzenia nowych połączeń.

\subsubsection{Reprezentacja mapy}
Mapa z przystankami jest reprezentowana jako ważony graf z biblioteki \lstinline{NetworkX}.

\subsubsection{Reprezentacja genotypu}
Genotyp składa się z listy linii autobusowych:
\begin{lstlisting}[language=Python]
class Genotype:
    def __init__(self, lines: list[Line]):
        self.lines = lines
\end{lstlisting}

\subsubsection{Reprezentacja linii}
Linia posiada następujące parametry:
\begin{enumerate}
    \item \lstinline{id} - id linii,
    \item \lstinline{stops} - przystanki, na których się zatrzymuje,
    \item \lstinline{edges} - wszystkie krawędzie, przez które linia przejeżdża,
    \item \lstinline{edge_color} - kolor linii; do reprezentacji graficznej,
    \item \lstinline{edge_style} - styl krawędzi linii; do reprezentacji graficznej,
\end{enumerate}


\begin{lstlisting}[language=Python]
class Line:
    def __init__(self, stops: list[int], best_paths):
        self.id = Line.get_next_id()
        # ordered list of stops
        self.stops = stops
        # list of edges on shortest paths between successive stops
        self.edges = []
\end{lstlisting}

Przystanki są uporządkowanym zbiorem wierzchołków z grafu. Każda linia jest jednoznacznie reprezentowana przez swoje przystanki $stops$.
Zbiór krawędzi należących do danej linii $egdes$ tworzymy w następujący sposób:
\begin{enumerate}
    \item Generujemy słownik $best_paths$ zawierający najkrótsze ścieżki od każdego do każdego wierzchołka. Używamy do tego funkcji $all_pairs_shortest_path()$ z biblioteki NetworkX
    \item Dla każdej pary kolejnych wierzchołków $u$ i $v$ w zbiorze $stops$ do zbioru $edges$ dodajemy wszystkie krawędzie leżące na najkrótszej ścieżce między $u$ i $v$
\end{enumerate}

\subsection{Rozwiązanie początkowe}
Na początku, chcąc się skupić na realizacji algorytmu, wygenerowaliśmy losowo sieć połączeń. Powstała ona przez wygenerowanie N punktów na płaszczyźnie, a następnie połączeniu ich między sobą z pewnym prawdopodobieństwem. Dawało to całkiem dobre rezultaty:
\img{map_seed_46}{Przykładowa wygenerowana mapa}

\subsection{Symulacja}
\begin{algorithm}[H]
    \caption{Symulacja}
    \begin{algorithmic}[1]
        \Function{Symuluj}{liczba\_pokoleń, x}
        \State populacja = \Call {populacja\_początkowa}{ }
        \State \Call {zapisz\_populację}{ }
        \ForTo{i}{0}{\text{liczba\_pokoleń} - 1}
        \State populacja = \Call{usun\_niedopuszczalne}{populacja}
        \State populacja\_dopasowanie = \Call{fitness}{populacja}

        \State populacja = \Call{funkcja\_przetrwania}{populacja, populacja\_dopasowanie}
        \State populacja\_nowa = \Call{nowa\_populacja}{populacja} \Comment{Tutaj zachodzą mutacje i krzyżowania}

        Co x epok:
        \State \Call {zapisz\_populację}{ }
        \EndFor
        \EndFunction
    \end{algorithmic}
\end{algorithm}

Powyżej przedstawiony został podstawowy silnik symulacji. W każdej epoce wykonuje on następujące kluczowe czynności:
\begin{itemize}
    \item Usuwa niedopuszczalne rozwiązania (linie bez przystanków, organizmy bez linii),
    \item Oblicza funkcję dopasowania,
    \item Uruchamia funkcję przetrwania, która likwiduje wybrane osobniki,
    \item Uruchamia funkcję nowej populacji, która dokonuje mutacji i krzyżowań.
\end{itemize}

Na tym poziomie nie definiujemy, co dana funkcja robi. Zostało to zrobione poniżej.

\subsection{Selekcja}
\label{sec:selection}
Wypróbowaliśmy wielu różnych metod selekcji nowych osobników:
\begin{enumerate}
    \item \lstinline{n_best_survive(n)} - pozostawia daną liczbę $n$ najlepszych osobników,
    \item \lstinline{n_best_and_m_random_survive(n, m)} - pozostawia $n$ najlepszych osobników, oraz $m$ losowych spośród pozostałych,
    \item \lstinline{n_best_and_m_worst_survive(n, m)} - pozostawia $n$ najlepszych i $m$ najgorszych osobników,
    \item \lstinline{exponentional_survival(n, lambda)} - pozostawia $n$ osobników w sposób losowy, ale zależny od uzyskanej wartości $fitness$ i zgodny z rozkładem wykładniczym z parametrem $lambda$,
    \item \lstinline{exponentional_survival_with_protection(best_protected, worst_protected, lambda)} - działa jak $exponentional\_survival(lambda, n)$, ale gwarantuje przeżycie $best\_protected$ najlepszym i $worst\_protected$ najgorszym osobnikom,
\end{enumerate}

\subsection{Mutacja}

\subsubsection{LineMutator}
Tworzy nowe mutacje dla danej linii.

Możliwe mutacje:

\begin{enumerate}
    \item \lstinline{rotation_to_right} - losuje spójny ciąg przystanków w linii i przesuwa je o zadaną (lub losową) liczbę pozycji
    \item \lstinline{cycle_rotation} - losuje pozycje przystanków w linii i przesuwa obecne na tych pozycjach przystanki o jedną pozycję w ramach wylosowanych pozycji
    \item \lstinline{invert} - odwraca kolejność przystanków, pomiędzy losowymi indeksami \lstinline{start} oraz \lstinline{end},
    \item \lstinline{erase_stops} - losowo usuwa zadaną liczbę przystanków z linii,
    \item \lstinline{add_stops} - losowo dodaje zadaną liczbę przystanków, spośród tych, które w linii nie występują
    \item \lstinline{replace_stops} - losowo zmienia zadaną liczbę przystanków z linii na inne. Nowe przystanki są wybierane z rozkładu jednostajnego lub wykładniczego gdzie przystanki bliższe do obecnego są bardziej prawdopodobne
\end{enumerate}

\subsubsection{GenotypeMutator}

Możliwe mutacje:

\begin{enumerate}
    \item \lstinline{erase_line(G)} - tworzy nowy genotyp, usuwając losową linię,
    \item \lstinline{create_line(G)} - tworzy nowy genotyp, dodając losowo wygenerowaną linię,
    \item \lstinline{split_line(G)} - tworzy nowy genotyp, rozdzielając losową, losową linię dwie różne.
    \item \lstinline{merge_lines(G)} - tworzy nowy genotyp, łącząc zadaną liczbę losowych linii. W zależności od wartości parametru linie mogą być łączone całościowo lub na poziomie pojedynczych przystanków
    \item \lstinline{cycle_stops_shift(G)} - tworzy nowy genotyp, ustawiając ciągi przystanków z linii obok siebie i wykonując \lstinline{cycle_rotation} na takim ciągu przystanków
\end{enumerate}

\subsection{Krzyżowanie}

\subsubsection{GenotypeCrosser}

\begin{enumerate}
    \item \lstinline{merge_genotypes(G1, G2)} - tworzy nowy genotyp poprzez wybranie losowych linii z genotypów G1 i G2
    \item \lstinline{cycle_stops_shift(G1, G2)} - najpierw wykonuje \lstinline{merge_genotypes(G1, G2)}, a następnie \lstinline{GenotypeMutator.cycle_stops_shift(G)}
    \item \lstinline{line_based_merge(G1, G2)} - dzieli każdą z linii z G1 i G2 na połowy i jedną z połów każdej linii łączy z połową linii z drugiego genotypu. Z 4 możliwych przypadków połączenia wybiera ten, w którym dystans pomiędzy połączonymi przystankami jest minimalny
\end{enumerate}

\subsection{Rozwiązania niedopuszczalne}

\subsubsection{Sanitizer}

\begin{enumerate}
    \item \lstinline{BasicSanitizer} - podstawowy sanitizer usuwający sąsiednie wystąpienia tego samego przystanku, linie o zerowej długości i osobników bez linii
    \item \lstinline{RejectingSanitizer(criterium_sanitizer: Sanitizer)} - sprawdza czy \lstinline{criterium_sanitizer} wprowadziłby jakiekolwiek zmiany w genotypie i jeżeli tak to go usuwa
\end{enumerate}

\section{Aplikacja}
% Aplikacja -- krótka dokumentacja użytkownika
% jak uruchomić
% charakterystyka danych
% funkcjonalność aplikacji
% w jaki sposób ustawić parametry instancji
% jak ustawić parametry algorytmu
% wynikii sposób ich interpretacji

Aby uruchomić aplikację należy zainstalować interpreter języka Python w wersji co najmniej 3.11. Należy także, przy pomocy programu \lstinline|pip| zainstalować wymagane biblioteki poleceniem:
\lstinputlisting{lst/rme/1deps}

Aplikacja składa się z 3 modułów:
\begin{itemize}
	\item \lstinline|main.py| -- główny moduł aplikacji do uruchamiania eksperymentów,
	\item \lstinline|grid_search.py| -- automatyczne wielordzeniowe przeszukiwanie hiperparametrów,
	\item \lstinline|experiments.ipynb| -- notatnik Jupyter do szybkiego testowania algorytmu dla grafów losowych lub dla grafu miasta Krakowa.
\end{itemize}

\subsection{Notatnik Jupyter}
Najłatwiejszą metodą uruchomienia aplikacji jest notatnik \lstinline|experiments.ipynb|. Pozwala on na szybkie uruchomienie algorytmu dla losowego grafu oraz dla grafu miasta Krakowa. Szybkie wyświetlanie wyników umożliwia weryfikację działania programu.

\subsection{Głowny moduł}
Plik \lstinline|main.py| służy jako przykład uruchomienia naszego algorytmu. Zawiera on funkcje \lstinline|run_simulation| przyjmującą następujące parametry:
\begin{itemize}
	\item \lstinline|G: Graph| - reprezentacja grafowa miasta,
	\item \lstinline|all_stops: list[int]| - Przystanki w mieście (np. wszystkie wierzchołki w G)
	\item \lstinline|best_paths| - słownik najkrótszych ścieżek pomiędzy każdymi dwoma przystankami w G
	\item \lstinline|no_of_generations: int| - Liczba pokoleń do symulowania
	\item \lstinline|report_every_n: int| - Zapisz/wyświetl wynik co N epok
	\item \lstinline|report_show: bool| - jeżeli Prawda to wyświetl wyniki co N epok, jeżeli Fałsz to zapisz wynik do pliku w katalogu \lstinline|results|
	\item \lstinline|simulation_params: SimulationParams| - opcjonalne parametry symulacji
\end{itemize}


\subsubsection{Ustawienie parametrów}
Aby ustawić parametry przykładowego uruchomienia należy odszukać zmienną \lstinline|params| w funkcji \lstinline|run_simulations|. Zapisane są tam wszystkie parametry algorytmu genetycznego.

\subsubsection{Uruchomienie}
Żeby uruchomić główny moduł w konsoli, należy wykonać (w katalogu głównym):
\lstinputlisting{lst/rme/2runmain}

Skrypt zapyta nas o wybór miasta:
\lstinputlisting{lst/rme/2mainex}
Po dokonaniu wyboru rozpoczną się obliczenia.

\paragraph{Wyniki}
Wyniki obliczeń głównego modułu zapisywane są w postaci rysunków w folderze \lstinline|results|. W konsoli na bieżąco wyświetlane są statystyki:
\lstinputlisting{lst/rme/2mainres}

\subsection{Poszukiwanie hiperparametrów}
Moduł \lstinline|grid_search.py| został zaprojektowany jako wielowątkowy moduł uruchamiany z konsoli (w celu np. uruchomienia na zdalnym serwerze) z zapisem parametrów do pliku. Aby go uruchomić należy w terminalu wydać polecenie (głowny katalog projektu):
\lstinputlisting{lst/rme/3rungrid}

\paragraph{Wyniki}
Wyniki zostaną zapisane w pliku \lstinline|gridsearch.csv| w katalogu \lstinline|results|. W konsoli wyświetlają się słowniki parametrów aktualnie testowanych.

\subsubsection{Ustawianie parametrów}
Aby ustawić parametry do sprawdzenia, należy edytować zmienną \lstinline|grid_search_params| w pliku \lstinline|grid_search.py|. Zawiera ona słownik list. Każdy hiperparametr ma swoją listę, w której zapisane są sprawdzane wartości tego parametru. Trzeba pamiętać że algorytm przeszukiwania sprawdza wszystkie możliwe kombinacje parametrów, więc sprawdzenie dużej przestrzeni może zająć znaczący czas.

Parametry algorytmu można podać również do konstruktora klasy \lstinline|SimulationEngine|.


\section{Eksperymenty}
Na naszym algorytmie przeprowadziliśmy szereg eksperymentów.

\subsection{Eksperymenty proste}
\imgsidebyside{test1/0}{Populacja 0\\ dopasowanie $-121.46$}{test1/3}{Populacja 3\\ dopasowanie $1.71$}
\imgsidebyside{test1/5}{Populacja 5\\ dopasowanie $11.08$}{test1/10}{Populacja 10\\ dopasowanie $14.40$}
Jak widzimy, już po 10 epokach sieć połączeń znacznie się wyklarowała. Funkcja dopasowania wzrosła znacząco od generacji 0 do 10.

\imgsidebyside{test1/20}{Populacja 20\\ dopasowanie $17.18$}{test1/100}{Populacja 100\\ dopasowanie $24.64$}
Sieć pokryła jeszcze więcej przystanków. Tempo wzrostu funkcji dopasowania zmalało.

\imgcustomsize{test1/plot}{Wykres funkcji dopasowania}{0.6}
Jak widać, rzeczywiście tempo dopasowywania się modelu znacznie spada w późniejszych etapach symulacji.

\subsection{Przeszukiwanie siatki hiper-parametrów}

Pierwszy eksperyment obejmował wszystkie funkcje przetrwania. Przeszukiwana przestrzeń parametrów:
\lstinputlisting{lst/hp}

Funkcje przetrwania (opisy funkcji w sekcji \ref{sec:selection}):
\begin{itemize}
	\item [0] \lstinline|n_best_survive(N // 4)|
	\item [1] \lstinline|n_best_survive(N // 8)|
	\item [2] \lstinline|n_best_and_m_random_survive(N // 4, N // 10)|
	\item [3] \lstinline|n_best_and_m_random_survive(N // 4, N // 20)|
\end{itemize}

\img{gs1/all}{Rozkład funkcji dopasowania dla pierwszego przeszukiwania siatki hiper-parametrów}

\imgsidebyside{gs1/1}{Rozkład dopasowania względem hiper-parametru}{gs1/2}{Rozkład dopasowania względem hiper-parametru}
\imgsidebyside{gs1/3}{Rozkład dopasowania względem hiper-parametru}{gs1/4}{Rozkład dopasowania względem hiper-parametru}
\imgsidebyside{gs1/5}{Rozkład dopasowania względem hiper-parametru}{gs1/6}{Rozkład dopasowania względem hiper-parametru}
\imgsidebyside{gs1/7}{Rozkład dopasowania względem hiper-parametru}{gs1/8}{Rozkład dopasowania względem hiper-parametru}
\imgsidebyside{gs1/9}{Rozkład dopasowania względem hiper-parametru}{gs1/10}{Rozkład dopasowania względem hiper-parametru}

Jak widać, niektóre rozkłady są lewoskośne, więc dalsze eksperymenty zawęziliśmy do wartości dla tych rozkładów. Jeżeli był to rozkład z parametrem 0.8 to wartości w dalszych eksperymentach to 0.5, 0.75, 0.9; dla parametru o wartości 0.2: 0.1, 0.25, 0.5. Dla rozkładów symetrycznych przyjęliśmy stałą wartość 0.5 (oprócz \lstinline|chance_merge_specimen| -- tutaj zostawiono duży rozrzut).

Zdecydowanie lepiej radzi sobie funkcja przetrwania 0 od 1 i analogicznie 3 od 2. W poniższym eksperymencie zostały porównane tylko 0 i 1 ale do pozostałych wrócono niżej.

Teraz przestrzeń parametrów wygląda następująco:
\lstinputlisting{lst/hp2}

\img{gs2/all}{Rozkład funkcji dopasowania dla drugiego przeszukiwania siatki hiper-parametrów -- zawężona przestrzeń parametrów}

Minimum wzrosło z 18.25 do 27.16, idziemy w dobrą stronę! Ale maksimum wzrosło tylko o 0.2.

\imgsidebyside{gs2/1}{Rozkład dopasowania względem hiper-parametru}{gs2/2}{Rozkład dopasowania względem hiper-parametru}
\imgsidebyside{gs2/3}{Rozkład dopasowania względem hiper-parametru}{gs2/4}{Rozkład dopasowania względem hiper-parametru}
\imgsidebyside{gs2/5}{Rozkład dopasowania względem hiper-parametru}{gs2/6}{Rozkład dopasowania względem hiper-parametru}
\imgcustomsize{gs2/7}{Rozkład dopasowania względem hiper-parametru}{0.5}

Najlepsze parametry z wykresów odczytano jako (najbardziej lewoskośny/najwięcej przypadków po prawej/najmniej po lewej):
\lstinputlisting{lst/hp3}

Maksymalne dopasowanie (35.77) osiągnięto dla
\lstinputlisting{lst/hp3_max}

Jedyna różnica w \lstinline|chance_erase_line|. Na wykresie kolor niebieski ($=0.1$) i pomarańczowy ($=0.25$) prawie się pokrywają.

Dla najlepszych parametrów graf miasta prezentuje się następująco:
\imgsidebyside{best_params_hp3}{Epoka 100, dopasowanie 34.90}{best_params_hp3_1000}{Epoka 1000, dopasowanie 41.96}

Nadal widoczne są patologiczne sytuacje. Np. wierzchołki 19 i 23 są połączone tylko między sobą.

\subsubsection{Dodatkowe operatory genetyczne}

Przetestowaliśmy dodatkowo (poprzednie parametry takie same jak powyżej):
\lstinputlisting{lst/hp4}

Funkcje przetrwania:
\begin{itemize}
	\item [1] \lstinline|n_best_survive(N // 8)|
	\item [4] \lstinline|n_best_and_m_random_survive(N // 8, N // 20)| -- połączenie 1 i 3 z poprzednich eksperymentów (najlepsze wyniki)
\end{itemize}

\img{gs3/all}{Rozkład funkcji dopasowania dla trzeciego przeszukiwania siatki hiper-parametrów -- nowe operatory}

Minimum spadło z 27.16 do 26.90, a maksimum z 35.77 do 34.72.

\imgsidebyside{gs3/1}{Rozkład dopasowania względem hiper-parametru}{gs3/2}{Rozkład dopasowania względem hiper-parametru}
\imgsidebyside{gs3/3}{Rozkład dopasowania względem hiper-parametru}{gs3/4}{Rozkład dopasowania względem hiper-parametru}
\imgsidebyside{gs3/5}{Rozkład dopasowania względem hiper-parametru}{gs3/6}{Rozkład dopasowania względem hiper-parametru}
\imgsidebyside{gs3/7}{Rozkład dopasowania względem hiper-parametru}{gs3/8}{Rozkład dopasowania względem hiper-parametru}
\imgsidebyside{gs3/9}{Rozkład dopasowania względem hiper-parametru}{gs3/10}{Rozkład dopasowania względem hiper-parametru}

Ponownie zawężony została przestrzeń przeszukiwań parametrów (zgodnie z zasadami z wcześniejszych eksperymentów):
\lstinputlisting{lst/hp5}

\img{gs4/all}{Rozkład funkcji dopasowania dla czwartego przeszukiwania siatki hiper-parametrów -- nowe operatory}

Minimum wzrosło z 32.34 do 26.90, a maksimum z 34.72 do 35.90 (przed implementacją nowych operatorów było to 35.77). Mamy nowy globalnie lepszy wynik.

\imgsidebyside{gs4/1}{Rozkład dopasowania względem hiper-parametru}{gs4/2}{Rozkład dopasowania względem hiper-parametru}
\imgsidebyside{gs4/3}{Rozkład dopasowania względem hiper-parametru}{gs4/4}{Rozkład dopasowania względem hiper-parametru}
\imgsidebyside{gs4/5}{Rozkład dopasowania względem hiper-parametru}{gs4/6}{Rozkład dopasowania względem hiper-parametru}
\imgsidebyside{gs4/7}{Rozkład dopasowania względem hiper-parametru}{gs4/8}{Rozkład dopasowania względem hiper-parametru}
\imgcustomsize{gs4/9}{Rozkład dopasowania względem hiper-parametru}{0.5}

Teraz całość najlepszych hiperparametrów wygląda następująco:
\lstinputlisting{lst/hp6}

\subsection{Eksperymenty z mapą Krakowa}
Udało nam się pobrać mapę Krakowa dzięki bibliotece OSMNX (OpenStreetMap NetworkX). Dane o ludności pobrano z msip.krakow.pl. Mapa z nałożoną punktacją wierzchołków (wg modelu) wygląda następująco:
\imgsidebyside{krk}{Mapa Krakowa z danymi o ludności}{krk1.png}{Wynik algorytmu}
Jaśniejszy kolor wierzchołka oznacza wyższą wartość punktową.
Po uruchomieniu naszego algorytmu uzyskaliśmy:
Ile przystanków ma ile linii:
\lstinputlisting{lst/krk1}

\section{Podsumowanie}

Problem optymalnego rozplanowania komunikacji miejskiej jest niezwykle skomplikowany. W naszej pracy rozważyliśmy jedynie jego uproszczoną wersję: rozłożenie linii, a i tak trudno jest o jednoznaczne wnioski.

Podczas pracy nad rozwiązaniem wielokrotnie generowaliśmy rozwiązania ekstremalne i niepoprawne np. pokrycie miasta liniami wyłącznie o długości 2 lub jedną linią długości 500. Problemy te wynikały z tego, że algorytm genetyczny próbował znaleźć słaby punkt w naszej funkcji zysku. Jeśli kara za utworzenie nowej linii była zbyt wysoka, optymalnym rozwiązaniem była jedna długa linia. Jeśli kara ta była za niska, a koszt podróży po krawędzi zbyt wysoki, lepsze okazywało się tworzenie setek linii o najkrótszej możliwej długości. Ostatecznie udało nam się tak dobrać funkcję zysku, że powstawały rozwiązania "naturalne" (zarówno krótkie jak i dłuższe linie).

Kolejnym krokiem w pracy nad rozplanowaniem komunikacji mogłoby być generowanie przykładowych rozkładów jazdy, które musiałby sprostać zmiennemu natężeniu podróżujących jak i ograniczonej liczbie pojazdów. Takie rozwiązanie stanowiłoby już bliski rzeczywistości model komunikacji miejskiej.

\section{Bibliografia}
\begin{enumerate}

\item\href{
https://docs.python.org/3/
}{Python 3.11}

\item\href{
https://networkx.org/documentation/stable/index.html
}{NetworkX}

\item\href{https://osmnx.readthedocs.io/en/stable/}{OSMNX}

\item\href{https://pl.wikipedia.org/wiki/Prawo_malej%C4%85cych_przychod%C3%B3w}{Prawo malejących przychodów (Wikipedia)}

\item\href{https://msip.krakow.pl}{Dane o zaludnieniu (Miejski System Informacji Przestrzennej)}
\end{enumerate}

\section{Podział pracy}
\begin{itemize}
    \item Jakub Kosmydel - 25\%
    \begin{itemize}
        \item implementacja algorytmu Grid Search do poszukiwania optymalnych hiperparametrów,
        \item wizualizacje eksperymentów,
        \item dokumentacja.
    \end{itemize}
    \item Norbert Morawski - 25\%
    \begin{itemize}
        \item przeszukiwanie przestrzeni hiperparametrów z użyciem Grid Search,
        \item zaimportowanie rzeczywistych danych przy użyciu OpenStreetMap,
        \item eksperymenty z optymalnymi parametrami,
        \item dokumentacja.
    \end{itemize}
    \item Przemysław Węglik - 25\%
    \begin{itemize}
        \item generacja grafów losowych,
        \item prace nad architekturą projektu (Simulation Engine),
        \item pierwsze eksperymenty.
    \end{itemize}
    \item Bartłomiej Wiśniewski - 25\%
    \begin{itemize}
        \item Stworzenie większości algorytmów mutujących i krzyżujących
        \item Stworzenie logiki sprowadzającą osobników do obrzaru rozwiązań dozwolonych (Sanitizery)
        \item Prace nad architekturą projektu (klasy Line i Genotype)
        \item Prace nad utrzymywaniem kodu (typing, rewizje i śledzenie tasków)
    \end{itemize}
\end{itemize}
\end{document}
