% !TeX spellcheck = pl_PL
\documentclass[12pt,a4paper,openright]{mwrep}
%\documentclass[12pt,a4paper,openright]{article}

% Przemkowe importy co nie działają
%\usepackage{lmodern}
%\usepackage[T1]{polski}
%
%\usepackage[a4paper,
%            tmargin=2cm,
%            bmargin=2cm,
%            lmargin=2cm,
%            rmargin=2cm,
%            bindingoffset=0cm]{geometry}
%
%\usepackage{tocloft}
%\usepackage{hyperref}
%
%\usepackage{amsmath}
%\usepackage{amssymb}
%\usepackage{siunitx}
%
%\usepackage{listings}
%
%\usepackage{graphicx}
%\usepackage{subfig}
%\usepackage{float}
%\usepackage{booktabs}

% moje działają
\usepackage{amssymb} % symbol kąta
\usepackage[polish]{babel} % polskie nazwy
\usepackage[T1]{fontenc} % polskie znaki
\usepackage[margin=1.0in]{geometry} % marginesy
\usepackage[utf8]{inputenc}
\usepackage{listingsutf8} % bloki kodu
\usepackage{color} % kolory
\usepackage{indentfirst} % wcięcie w pierwszej linii paragrafu
\usepackage{graphicx} % obrazy
\usepackage{float} % dla image [H]
\usepackage{amsmath,amsthm,amssymb,mathtools} % matematyka dowód
\usepackage{changepage} % matematyka dowód
\usepackage{siunitx} % wyrównanie do kropki
\usepackage{makecell} % wyrównania nagłówków
%\usepackage{enumitem} % wyrównania nagłówków
\usepackage{tikz} % zbocza
\usetikzlibrary{decorations.markings}
\usepackage{hyperref} % bez obwódek wokół linków

\hypersetup{
    colorlinks,
    citecolor=black,
    filecolor=black,
    linkcolor=black,
    urlcolor=black
}

\newtheorem{definition}{Def}

\begin{document}

\title{
Badania operacyjne\\
Projekt\\
}

\author{\\Jakub Kosmydel\\Norbert Morawski
\\Bartłomiej Wiśniewski\\Przemysław Węglik}

\date{\today}

\maketitle

\chapter{Wstęp}
Celem naszego projektu jest znalezienie optymalnych tras linii dla autobusów, aby maksymalizować liczbę pasażerów, przy minimalnej liczbie linii autobusowych. Aby to osiągnąć, wykorzystywane są algorytmy genetyczne -  algorytmy przeszukujące przestrzeń rozwiązań, które opierają się na procesie działania mechanizmu dziedziczenia biologicznego.

W systemie założono, że pozycje oraz popularność przystanków są z góry ustalone. Stosowanie algorytmów genetycznych pozwoliło na wygenerowanie zestawu najlepszych połączeń autobusowych, które można skonfigurować dla lepszego wykorzystania zasobów oraz zwiększenie korzyści z transportu publicznego dla pasażerów.

\chapter{Opis zagadnienia}

\section{Sformułowanie problemu}
Naszym celem w projekcie jest zaprojektowanie sieci linii autobusowych pokrywającej dany obszar miejski, który już posiada sieć przystanków autobusowych. Linie te, powinny mieć możliwość obsłużenia jak największej liczby pasażerów, tworząc jak najmniej postojów oraz zatrzymując się na jak najmniejszej liczbie przystanków.

\section{Model matematyczny}

\subsection{Założenia}

\begin{enumerate}
	\item Przystanki mają jakąś ilość punktów w zależności od gęstości zaludnienia i ciekawych punktów
	\begin{itemize}
		\item Dla każdego przystanku obliczyć wartość ludzi jako % TODO function
		\item Głównym punktom w Krakowie (D17 itp) nadać wartość punktową (jakoś)
		\item Dla każdego przystanku wyliczyć wartość obiektową tak jak i wartość ludzi
	\end{itemize}
	\item Rozkładamy linie komunikacyjne po mieście tak, by maksymalizować sumę zebranych punktów przez wszystkie linie
	\item Wprowadzamy koszt dla linii jest jednostkowy + koszt ścieżki w grafie po której jedzie
	\item Punkty dzielą się między linie w następujący sposób (1 linia - 100\%, 2 linie - 66\% każda, 3 linie - każda po 50\% etc.) Fajnie jakby to zbiegało do jakieś liczby, może np. do 2?
	\item Maksymalizujemy sumę punktów zebranych przez wszystkie linie
\end{enumerate}

\subsection{Dane}
	\begin{enumerate}
		\item $n$ - liczba linii
		\item $m$ - liczba przystanków
	\end{enumerate}

\subsubsection{Graf}

	\begin{enumerate}
		\item Wierzchołki to skrzyżowania (0 punktów) + istniejące przystanki (punkty wg wzoru z pkt 1 - “ile ludzi chce jechać z niego” i pobliskie atrakcje/ważne miejsca)
		\item $p(j)$ - surowa wartość punktowa przystanku
		\begin{itemize}
			\item $p(j) = \sum_{i=0}^{n-1} \frac{w_{j, i}}{f(d_j, i)}$ gdzie $w_{j, i}$ to wartość obiektu (np. liczba mieszkańców bloku) a $d_{j,i}$ to odległość tego bloku od przystanku, f – funkcja skalująca
			\item Funkcja liczona dla danego przystanku $j$
		\end{itemize}
		\item Krawędzie to ulice między skrzyżowaniami rozdzielone przez przystanki
		\item Koszt krawędzi to odległość między punktami
	\end{enumerate}










\subsection{Szukane}
	$x_{i,j}$ - czy linia $i$ zatrzymuje się na przystanku $j$, gdzie:
	\begin{enumerate}
		\item $i \in \left[ 0, n-1 \right]$
		\item $j \in \left[ 0, m-1 \right]$
	\end{enumerate}

\subsection{Hiperparametry}
	\begin{enumerate}
		\item $\alpha$ - koszt zatrzymania się na przystanku,
		\item $\beta$ - koszt nowej linii,
		\item $R$ - hiper parametr zbiegania.
	\end{enumerate}



$l_j = \sum_{i=0}^{n-1} x_{i, j}$\\
$p_{i, j} = \frac{p_j \cdot (1+\frac{R}{l_j})^l_j}{l_j}$ -- ile punktów linia $i$ uzyskuje z przystanku $j$

$S_{i}$ -- długość ścieżki linii $i$ w grafie

$f(x) = \sum_{i=0}^{n-1} \sum_{j=0}^{m-1} x_{i,j} * (p_{i,j}-\alpha)-S_{i}-\beta$


\chapter{Opis algorytmów}

\section{Mutacja}

\subsection{LineMutator}

\subsection{GenotypeMutator}

\section{Krzyżowanie}


\chapter{Aplikacja}

\chapter{Eksperymenty}

\chapter{Podsumomwanie}

\end{document}
